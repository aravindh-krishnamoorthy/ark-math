\section{Inverse Cholesky Factor of a positive definite Hermitian symmetric Toeplitz matrix using Durbin recursions}
\label{sec:001_inverse}

Let $\boldsymbol{T} \in \mathbb{C}^{n \times n}$ be a positive-definite Hermitian symmetric Toeplitz matrix defined with elements $t_{ij} = r_{i-j}$ for a set of scalars $r_{-n+1}, \cdots, r_0, \cdots, r_{n-1}$, where $r_{-t} = r_t^*$. Without loss of generality, assume $r_0 = 1$.

We have the Cholesky factor $\boldsymbol{R}$ of the matrix $\boldsymbol{T}$ such that $\boldsymbol{R}^{\mathrm{H}} \boldsymbol{R} = \boldsymbol{T}$. The inverse Cholesky factor is then given by $\boldsymbol{R}^{-1}$. A closely related decomposition known as the LDL decomposition is given as follows: $\boldsymbol{\Lambda} = \text{diag}(\boldsymbol{R})$, $\boldsymbol{L} = \boldsymbol{R}^{\mathrm{H}} \boldsymbol{\Lambda}^{-1}$, and therefore, $\boldsymbol{T} = \boldsymbol{L} \boldsymbol{\Lambda}^2 \boldsymbol{L}^{\mathrm{H}} = \boldsymbol{L} \boldsymbol{D} \boldsymbol{L}^{\mathrm{H}}$. Matrix $\boldsymbol{L}$ has ones along the principal diagonal. Let $\boldsymbol{\Sigma} = \boldsymbol{L}\boldsymbol{D}$.

We have $\boldsymbol{T} \boldsymbol{L}^{-\mathrm{H}} = \boldsymbol{\Sigma}$, where $\boldsymbol{L}^{-\mathrm{H}}$ is the Hermitian transpose of the inverse with ones along the diagonal. As $\boldsymbol{\Sigma}$ is lower triangular, we can solve for the strictly upper triangular parts of $\boldsymbol{L}^{-\mathrm{H}}$ as the R.H.S. is known to be zero.

For the $k$-th column of $\boldsymbol{L}^{-\mathrm{H}}$ we have the first $k-1$ elements $\boldsymbol{z}_{k-1}$, upon simplification, as:
$$\boldsymbol{T}_{k-1} \boldsymbol{z}_{k-1} = - \begin{bmatrix} r_{k+1} \\ \vdots \\ r_1 \end{bmatrix}$$

For Hermitian symmetric Toeplitz matrix $\boldsymbol{T}$ we have $\boldsymbol{T} = \boldsymbol{E}_n \boldsymbol{T}^* \boldsymbol{E}_n$, where $\boldsymbol{T}^*$ is complex conjugate of the matrix and $\boldsymbol{E}_n$ is the $n \times n$ exchange matrix. Using this property, and setting $\boldsymbol{y}_{k-1} = \boldsymbol{E}_k \boldsymbol{z}^*_{k-1}$ we have:

$$\boldsymbol{T}_{k-1} \boldsymbol{y}_{k-1} = - \begin{bmatrix} r^*_1 \\ \vdots \\ r^*_{k+1} \end{bmatrix}$$,

which may be solved efficiently using the Durbin iterations \cite{Golub2012}. The algorithm for finding $\boldsymbol{W} = \boldsymbol{R}^{-1} = [w_{i,j}]$ is given as follows:

\begin{algorithm}[H]
\caption{Algorithm for finding $\boldsymbol{W}.$}
\begin{algorithmic}[1]
	\State{$y_1 = -r^*_1; \enspace \alpha = -r^*_1; \enspace \boldsymbol{W} = \boldsymbol{I}_n$}
	\State{$\beta = 1 - |\alpha|^2$}
	\State{$w_{1,2} = y^*_1$}
	\State{$\boldsymbol{w}_{:,2} = \frac{\boldsymbol{w}_{:,2}}{\sqrt\beta}$}
	\For{$k = 1:n-2$}
		\State{$\alpha = -(r^*_{k+1} + \frac{1}{\beta}\boldsymbol{r}^{\mathrm{H}}_{k:-1:1} \boldsymbol{y}_{1:k})$}
		\State{$\boldsymbol{y}_{1:k} = \boldsymbol{y}_{1:k} + \alpha \boldsymbol{y}^*_{k:-1:1}$}
		\State{$y_{k+1} = \alpha$}
		\State{$\beta = (1 - |\alpha|^2) \beta$}
		\State{$\boldsymbol{w}_{1:k+1,k+2} = \boldsymbol{y}^*_{k+1:-1:1}$}
		\State{$\boldsymbol{w}_{:,k+2} = \frac{\boldsymbol{w}_{:,k+2}}{\sqrt\beta}$}
	\EndFor
\end{algorithmic}
\end{algorithm}

The algorithm requires $2n^2$ flops \cite[Algorithm 4.7.1]{Golub2012} for the Durbin iterations, $n^2/2$ flops for scaling with $1/\sqrt\beta$ and $n$ inverse square root computations; and may be easily adapted to find $\boldsymbol{L}^{-1}$ and $\boldsymbol{D}$ of the LDL decomposition. A MATLAB reference code is available in \cite{KrishnamoorthyMathWorks2015} as well as reproduced below.

\subsection{MATLAB Reference Code}

\begin{lstlisting}[language=MATLAB,numbers=none]
function R = invchol_durbin(T)
% R = invchol_durbin(T)
% Finds the inverse of the Cholesky factor of the positive definite
% Hermitian symmetric Toeplitz matrix T (N>=2) using Durbin recursions [1].
%
% Aravindh Krishnamoorthy, aravindh.krishnamoorthy@fau.de, 04-Sep-2015.
% Released under the 2-clause BSD license.
%
% [1] Gene H. Golub, Charles F. Van Loan, Matrix Computations, Third
% Edition, Algorithm 4.7.1 (Durbin).

% Exchange matrix of order k
E = @(k) rot90(eye(k)) ;

% Normalise matrix
ts = T(1,1) ;
T = T/ts ;
t = transpose(T(1,:)) ;

%% Durbin recursions [1]
% Setup variables
N = size(T,1) ;
r = t(2:end) ;
y = zeros(N,1) ;
R = eye(N) ;

% Initial values
y(1) = -conj(r(1)) ;
b    = 1 ;
a    = -conj(r(1)) ;
% Update 'b'
b    = (1-abs(a)^2)*b ;

% Column 2
R(1,2) = conj(y(1)) ;
R(:,2) = R(:,2)/sqrt(b) ;
% Rest of the columns
for k=1:N-2
	% Update 'a' and the result
	a = -(conj(r(k+1)) + ctranspose(r(k:-1:1))*y(1:k))/b ;
	y(1:k) = y(1:k) + a*conj(y(k:-1:1)) ;
	y(k+1) = a ;
	% Update 'b'
	b = (1-abs(a)^2)*b ;
	% Write out the result
	R(1:k+1,k+2) = E(k+1)*conj(y(1:k+1)) ;
	R(:,k+2) = R(:,k+2)/sqrt(b) ;
end

% Scale back the result
R = R/sqrt(ts) ;
\end{lstlisting}

\subsection{Version History}
\begin{enumerate}
	\item \emph{First published: 4th Sep. 2015 on aravindhk-math.blogspot.com}
	\item \emph{Modified: 16th Dec. 2023 -- Style updates for \LaTeX}
\end{enumerate}
