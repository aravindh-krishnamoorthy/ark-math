\section{Complex-valued Durbin Algorithm with MATLAB code}

\emph{This post is a rehash of my earlier post in Section \ref{sec:001_inverse} where the complex-valued Durbin's algorithm was used indirectly for matrix inversion. Here, the algorithm is written down separately as it may be useful for some to have it ``out of the box.''}

The complex-valued versions of Durbin's and Levinson's algorithms may be arrived upon by straightforward application of description in \cite{Golub2012} sec. 4.7.2 ``Solving the Yule-Walker Equations'', and 4.7.3 ``The General Right Hand Side Problem.''

Let $r_0 = 1, r_1, r_2, \cdots, r_n \in \mathbb{C}$ and $\boldsymbol{T} \in \mathbb{C}^{\text{N}\times\text{N}}$ be a positive definite Hermitian symmetric Toeplitz matrix constructed with the scalars $r_0, r_1, \cdots, r_{n-1}$ (see example below), Durbin's algorithm solves the system $\boldsymbol{T} \boldsymbol{y} = - \begin{bmatrix}r_1 & r_2 & \vdots & r_n\end{bmatrix}^\mathrm{T}.$ The algorithm is as follows.

\begin{algorithm}[H]
\caption{Algorithm for finding $\boldsymbol{y}.$}
\begin{algorithmic}[1]
\State{$z(1) = -r_1^*; \beta=1; \alpha=-r_1^*$}
\For{$k = 1:n-1$}
\State{$\beta = \beta (1-|\alpha|^2)$}
\State{$\alpha = \frac{-1}{\beta}(r^*(k+1) - r(k:-1:1)^H z(1:k))$}
\State{$z(1:k) = z(1:k) + \alpha z^*(k:-1:1)$}
\State{$z(k+1) = \alpha$}
\EndFor
\State{$y = z^*$}
\end{algorithmic}
\end{algorithm}

\subsection{MATLAB Reference Code}
\begin{lstlisting}[language=MATLAB,numbers=none]
function Y = durbin(R)
% Y = durbin(R)
% where R is a vector with r0, r1, r2, ..., rn.
% Solves RY = -K, where R is the NxN symmetric Toeplitz matrix 
% with elements r0, ..., r(n-1) and K is a vector with elements
% r1, ..., rn.
%

R = R/R(1,1) ;
N = length(R)-1 ;
R = R(2:end) ;
Z = zeros(N,1) ;

Z(1) = -conj(R(1)) ;
A = -conj(R(1)) ;
B = 1 ;

for k=1:N-1
B = (1-abs(A)^2)*B ;
A = -(conj(R(k+1)) + R(k:-1:1)'*Z(1:k))/B ;
Z(1:k) = Z(1:k) + A*conj(Z(k:-1:1)) ;
Z(k+1) = A ;
end
Y = conj(Z) ;
\end{lstlisting}

\subsection{MATLAB Test Code}
\begin{lstlisting}[language=MATLAB,numbers=none]
>> r = complex(rand(10,1),rand(10,1)) ;
>> r(1) = 1 ;
>> T = toeplitz(r(1:9),conj(r(1:9))) ;
>> k = r(2:end) ;
>> y1 = inv(T)*-k ;
>> norm(y1-durbin(r))
ans = 1.0709e-10
\end{lstlisting}

\subsection{Version History}
\begin{enumerate}
	\item \emph{First published: 30th Jan. 2016 on aravindhk-math.blogspot.com}
	\item \emph{Modified: 17th Dec. 2023 -- Style updates for \LaTeX}
\end{enumerate}

