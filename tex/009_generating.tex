\section{Generating and Sampling Two Signed Bernoulli Random Variables with Given Correlation}

The following first constructs two signed $\{-1, +1\}$ Bernoulli variables $Y_1,$ $Y_2$ such that $p(Y_1 = -1) = p_1,$ $p(Y_2 = -1) = p2,$ and $E[Y_1Y_2] = r.$ Constraints: $0 < p1,p2 < 1, -1 < r < 1.$ 

Next, it draws $N$ realizations of the two variables in $Y(m,:), m = 1,2.$ The sample pdf and cross correlation converge as $N \to \infty.$

Example: generate $N$ realizations of variables with mean $0,$ variance $1,$ and correlation $0.3:$ Y = signed\_bernoulli2(N,0.5,0.5,0.3).

DOWNLOAD: The full program can be downloaded at: signed\_bernoulli2(N,p1,p2,r) - File Exchange - MATLAB Central (mathworks.com)

\begin{verbatim}
% Resolve the joint pdf given by
%                /  p(1) if (+1, +1)
% p(Y_1, Y_2) =  |  p(2) if (+1, -1)
%                |  p(3) if (-1, +1)
%                \  p(4) if (-1, -1)
%
% Conditions (i.e. how matrices A and b are generated):
% row 1: p(1) + p(2) + p(3) + p(4) = 1
% row 2: p(3) + p(4) = p1   -> marginal for Y_1
% row 3: p(2) + p(4) = p2   -> marginal for Y_2
% row 4: p(1) - p(2) - p(3) + p(4) = r   -> cross correlation E[Y_1 Y_2]
A = [1,1,1,1;0,0,1,1;0,1,0,1;1,-1,-1,1] ;
b = [1;p1;p2;r] ;
p = pinv(A)*b ;
assert(all(p >= 0), sprintf(['The given p1=%f and p2=%f cannot result in '...
' a correlation r=%f with the given algorithm.\np1=p2=0.5 is a safe value.'],...
p1,p2,r)) ;
% Now that we have the joint pdf in "p" we can draw from this distribution
% as follows (based on conditional pdfs and inverse transform sampling):
% 1. Draw samples of Y_1 from Signed_Bernoulli(p1) in a standard fashion using
%    an auxiliary uniformly distributed variable U_1 = U(1,:).
% 2. If the n-th sample of Y_1 is "-1", then draw for Y_2 from the conditional pdf
%    p(Y_2|Y_1 = -1) using a second axiliary uniformly distributed
%    variable U_2 = U(2,:).
% 2. If the n-th sample of Y_1 is "+1", then draw for Y_2 from the conditional pdf
%    p(Y_2|Y_1 = +1) using the second axiliary uniformly distributed
%    variable U_2 = U(2,:).
% The conditional pdfs are given as follows:
% p(Y_2|Y_1 = -1) = p(4)/p1 for -1, and 1-(p(4)/p1) for +1.
% p(Y_2|Y_1 = +1) = p(2)/(1-p1) for -1, and 1-(p(2)/(1-p1)) for +1.
%
% Note: for the rather straightforward proofs for the above steps, please contact
% the author.

% Auxiliary variable U
U = rand(2,N) ;
% Draw samples from p(Y_1)
Y(1,:) = (-1).^(U(1,:) < p1) ;
% Indices to help choose the conditional pdf
I1 = Y(1,:) == -1 ;
I2 = Y(1,:) == +1 ;
% Draw samples from the conditional pdfs
Y(2,I1) = (-1).^(U(2,I1) < p(4)/p1) ;
Y(2,I2) = (-1).^(U(2,I2) < p(2)/(1-p1)) ;
\end{verbatim}

\emph{First published: 23rd Nov. 2017 on aravindhk-math.blogspot.com}
